% Questo file contiene la composizione del frontmatter composto da title page + abstract + indici + simboli

%----------------------------------------------------------------------------------------
%	TITLE PAGE
%----------------------------------------------------------------------------------------
\begin{titlepage}
    % Contenuto centrato per logo e titolo
    \begin{center}
        \vspace{1cm}
        \includegraphics[width=7cm]{img/logo_unifi.png} \\[4cm] % Logo
        
        {\huge \textbf{Titolo}}\\[0.5cm]
        
        {\Large Sottotitolo}\\[1cm]
        
        {\large Corso di laurea magistrale in Ingegneria Civile}\\[0.5cm]
        {\large Nome del corso}\\
    \end{center}

    \vfill

    % Contenuto allineato a sx per info studenti e docenti/revisori
    \begin{flushleft}
        \textbf{Studente/i:} nome cognome 1, nome cognome 2 \\ 
        \textbf{Docente/i:} Prof. Ing. nome cognome\\ 
        \textbf{Revisore/i:} Prof. Ing. nome cognome \\[4cm]
    \end{flushleft}

    \vfill

    \begin{center}
	    \today % Per mostrare la data di compilazione
	\end{center}

    % Immagine di bg in basso a dx --> Da commentare se non usata
    \begin{textblock}{2}(8.8, 10)  % Adattare (x, y) per variare posizione
        \includegraphics[width=14cm]{img/logo_unifi_bg.png}
    \end{textblock}
\end{titlepage}

\newpage

%----------------------------------------------------------------------------------------
%	ABSTRACT
%----------------------------------------------------------------------------------------
% Commenta la sezione se non vuoi usare l'abstract
\begin{abstract}

Your abstract.

\end{abstract}
\newpage

%----------------------------------------------------------------------------------------
%	INDICI
%----------------------------------------------------------------------------------------
\pagestyle{plain}     % Plain style per tutte le pagine indice
\pagenumbering{Roman} % Numerazione romana (i, ii, iii, iv...) per le pagine indice

% Indice
\tableofcontents
\newpage

% Indice figure
\listoffigures
\newpage

% Indice tabelle
\listoftables
\newpage

%----------------------------------------------------------------------------------------
%	SIMBOLI
%----------------------------------------------------------------------------------------

% Lista dei simboli
\chapter*{Simboli}  % L'asterisco serve a non numerare il capitolo

\begin{longtable}{lll} % Long table per i simboli

%Symbol & Name & [Unit] \\
$\gamma_M$ & Peso di volume & $\left[ \frac{kN}{m^3} \right]$ \\
$G$ & Modulo elastico di taglio &  \\
$G_0$ & Modulo elastico di taglio iniziale &  \\

\end{longtable}

\newpage
\pagestyle{fancy}      % Riapplica header e footer da qui in poi
\pagenumbering{arabic} % Numerazione pagine normale (1,2,3...)
