% Questo file contiene tutte le info sui pacchetti utilizzati, le impostazioni base del documento, della sua geometria e la formattazione di titoli, header e footer

%----------------------------------------------------------------------------------------
%	PACCHETTI
%----------------------------------------------------------------------------------------
% Pacchetti necessari per il template
\usepackage[utf8]{}             % Input encoding
\usepackage{graphicx}           % Per le immagini
\usepackage{amsmath}            % Per equazioni e simboli mateamtici
\usepackage{amssymb}            % Altri simboli matematici
\usepackage{geometry}           % Per cambiare geometria
\usepackage[colorlinks=false]{hyperref}  % Hyperlinks non colorati (solo rettangolo)
\usepackage{fancyhdr}           % Per modificare header e footer
\usepackage{setspace}           % Line spacing
\usepackage{titlesec}           % Formattazione sezioni e capitoli
\usepackage{longtable}          % Per tabelle che possono splittarsi tra due pagine
\usepackage{float}              % Per posizionare le tabelle nel punto dove si vuole con [H]
\usepackage[absolute,overlay]{textpos} % Per posizionamento assoluto (immagine bg title page)
\usepackage{transparent}        % Per abilitare la trasparenza
\usepackage{fontspec}           % Per cambiare il font principale -> va usato LuaLaTex compiler
\usepackage{appendix}           % Per inserire le appendici
\usepackage{etoolbox}           % Per rimuovere indentazione dopo figure e tabelle

%----------------------------------------------------------------------------------------
%	IMPOSTAZIONI DOCUMENTO 
%----------------------------------------------------------------------------------------
% Impostazioni lingua
\usepackage[italian]{babel}

% Selezione del font --> LuaLaTex compiler (commentare per usare altro compiler)
\setmainfont{Georgia}  % Font per il corpo del testo

% Dimensioni di pagina e margini
\geometry{top=2cm, bottom=2cm, left=3cm, right=3cm}
\onehalfspacing % Line spacing settata a 1.5

\setcounter{secnumdepth}{3}  % Serve a numerare anche le subsubsection
\setcounter{tocdepth}{2}     % Serve a non mettere le subsubsection nell'indice

% Per levare l'indentazione del testo dopo immagini e tabelle (se non si va a capo)
\AfterEndEnvironment{table}{\noindent}
\AfterEndEnvironment{figure}{\noindent}

%----------------------------------------------------------------------------------------
%	HEADER E FOOTER
%----------------------------------------------------------------------------------------
% Per prevenire la trasformazione automatica del testo di header e footer in uppercase
\renewcommand{\chaptermark}[1]{\markboth{\thechapter.\ #1}{}} % Per i capitoli
\renewcommand{\sectionmark}[1]{\markright{\ #1}} % Per le sezioni
% Si definisce le altezze minime dell'header per evitare errori
\setlength{\headheight}{15pt}
\addtolength{\topmargin}{-2.5pt}

% Per cambiare header e footer delle pagine normali
\pagestyle{fancy}
\fancyhf{}   
\fancyhead[L]{\leftmark}
\fancyhead[R]{\rightmark}
\fancyfoot[C]{\thepage}

% Decommentare per cambiare header e footer delle pagine capitolo e dell'indice (plain style)
% \fancypagestyle{plain}{%
% \fancyhead{}
% \fancyfoot{}
% \fancyhead[R]{\thepage}
% }

%----------------------------------------------------------------------------------------
%	FORMATTAZIONE TITOLI SEZIONI E CAPITOLI
%----------------------------------------------------------------------------------------
% Decommentare per rimuovere la scritta "capitolo n" prima del capitolo
% \titleformat{\chapter}[hang]{\bfseries\Huge}{\thechapter.}{2pc}{}

% Formattazione scritte sezione e sottosezione
\titleformat{\section}[hang]{\bfseries\Large}{\thesection}{1pc}{}
\titleformat{\subsection}[hang]{\bfseries\large}{\thesubsection}{1pc}{}